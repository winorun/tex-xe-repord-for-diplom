\addcontentsline{toc}{chapter}{Приложение}
\chapter{Исходные коды и диаграммы программ}
\section{Основной аналог - Biharmon2 измененная}
%\lstinputlisting[language=C++, firstline=0]{source.cpp}
\begin{figure}[hp]
\begin{small}
\begin{tikzpicture}[
                     node distance=2cm,
                     text height=1.5ex,
                     auto,
					 text width=2.7cm,
					 align=flush center
                     %text depth=.25ex
]
\node[format,rectangle,minimum size=6mm,rounded corners=3mm]  (start) {Start};
\node[format,below of=start]  (p1) {$N,U,Disp$\\$ k=0$};
\node[format,diamond,text width=1.5cm,node distance=2.5cm,aspect=2,below of=p1] (if1) {$K<N$};
\path[->] (start)  edge (p1);
\path[->] (p1)  edge (if1);


\node[format,node distance=4.5cm,text width=4cm,right of=if1]  (p2) {$Disp-=U^2$\\$ Disp=\sqrt{\frac{|Disp|}{N}}$};
\path[->] (if1)  edge node[near start] {no} (p2);
\node[format,below of=p2,text width=2cm]  (p3) {Print all};
\path[->] (p2)  edge (p3);
\node[format,rectangle,minimum size=6mm,rounded corners=3mm,below of=p3]  (stop) {End};
\path[->] (p3)  edge (stop);

\node[format,below of=if1]  (p4) {$ k++$};
\path[->] (p4)  edge (if1);

\node[format,left of=if1,node distance=5cm]  (p5) {init $x,y$\\$S,S_1=0 $};
\path[->] (if1)  edge node[near start] {yes} (p5);

\node[format,diamond,text width=2.5cm,node distance=4cm,aspect=2,below of=p5] (if2){ boundary(x,y) };
\path[->] (p5)  edge (if2);

\node[format,right of=if2,node distance=5cm,text width=5cm]  (p6) {$S += \frac{S_1}{4}\phi_1(x,y)+\phi_0(x,y)$\\$U+=\frac{S}{N}$\\$Disp+=\frac{S^2}{N}$};
\path[->] (p6)  edge (p4);
\path[->] (if2) edge node[near start] {no} (p6);

\node[format,below of=if2]  (p7) {$ d=diam(x,y)$};

\node[format,below of=p7]  (p81) {$\alpha=DRAND$\\ $\omega_1=\cos 2\pi \alpha$\\$\omega_2=\sin 2\pi \alpha$};
\node[format,below of=p81]  (p82) {$\alpha=DRAND$\\ $\omega_3=\cos 2\pi \alpha$\\$\omega_4=\sin 2\pi \alpha$};
\path[->] (if2) edge node[near start] {yes} (p7);
\path[->] (p7) edge (p81);
\path[->] (p81) edge (p82);


\node[format,right of=p82,node distance=5cm]  (p9) {$\alpha_1=DRAND$\\ $\alpha_2=DRAND$};

\node[format,diamond,text width=3cm,aspect=2,below of=p9] (if3){ $\alpha_2>(4\alpha\log\alpha_1)$ };

\path[->] (p82) edge (p9);
\path[->] (p9) edge (if3);
\path[->,draw] (if3) -- node[near start] {yes} ++(3,0) |- (p9);

\node[format,below of=if3]  (p10) {$\nu=\alpha_1d$};
\node[format,below of=p10,text width=8cm]  (p11) {$S+=S_1-d^2\frac{d^2-\nu^2-\nu^2\log{\frac{d}{\nu}}}{\log{\frac{d}{\nu}}}*\frac{g(x+\nu\omega_1,y+\nu\omega_2)}{16}$};
\node[format,below of=p11]  (p12) {$S_1-=d^2$\\$x+=\omega_1d$\\$y+=\omega_2d$};
\path[->] (if3) edge node[near start] {no} (p10);
\path[->] (p10) edge (p11);
\path[->] (p11) edge (p12);
\path[->,draw] (p12) -- ++(-8,0)  |- (if2);
\end{tikzpicture}

\end{small}
\caption{Принцип действия программы}
\end{figure}
\chapter{Справка}
\section{Исполняемая программа}
biharmon -l |[ -d ]| -h\\
biharmon -l |[ -d ]| -h [filename]\\
- l	Latex-file - происходит создание файла report.tex\\
- d	Display - вывод данных на дисплей\\
- h	html-file - происходит создание файла report.html\\
Если файл существует к имени добавляется индекс. \\

filename - файл данных;

Приоритет -l, -h, -d \\
\section{Библиотека}
Открытые методы класса\\
init(int argc,char *args[])\\
setFunPhi();\\

\section{Файл данных}
Построчный.\\
X Y [SectionName]
\chapter{Выходные данные}
\section{tex,html}
\begin{center}
Отчет \\
N=$10^{4} $ \\
\begin{tabular}{|c|c|c|c|c|}
\hline
\textbf{$x,y$} & {\textbf{$h$}}& {\textbf{$u(r)$}}& {\textbf{$u^{~}(r)$}}& {\textbf{$|u(r)|$}}\\ \hline
\textbf{$0.5,0.5$} & {\textbf{$0.1$}}& {\textbf{$0.22$}}& {\textbf{$0.22$}}& {\textbf{$0.0004$}}\\ \hline
\end{tabular}
\end{center}
\chapter{Тезаурус}
API - Интерфейс программирования приложений (иногда интерфейс прикладного программирования) (англ. application programming interface) — набор готовых классов, процедур, функций, структур и констант, предоставляемых приложением (библиотекой, сервисом) для использования во внешних программных продуктах. 

MPI - Message Passing Interface (интерфейс передачи сообщений) — API для передачи информации, который позволяет обмениваться сообщениями между процессами, выполняющими одну задачу.
