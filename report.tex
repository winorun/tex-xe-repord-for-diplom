\documentclass[unicode, 12pt, a4paper, oneside, fleqn]{report}          
%edu.ci.nsu.ru ghtlbgkjvyfz ghfrnbrf
%rolcon@mail.ru
\usepackage{polyglossia}  %% подключает пакет многоязыковой вёрстки
\setmainfont[Mapping=tex-text]{Times New Roman}%%% Ставим Times New Roman - как основной шрифт
\setmonofont[Scale=MatchLowercase]{Courier New}%%% Courier New - для моноширного текста
\defaultfontfeatures{Mapping=tex-text}%%% Для того чтобы работали стандартные сочетания символов ---, --, << >> и т.п.
\setdefaultlanguage{russian}%%% Для работы с русскими текстами (расстановки переносов)
\newfontfamily\russianfont{Times New Roman}

\usepackage{geometry} % Меняем поля страницы
\geometry{left=2.5cm}% левое поле
\geometry{right=1.5cm}% правое поле
\geometry{top=1.0cm}% верхнее поле
\geometry{bottom=2cm}% нижнее поле

\usepackage{amssymb,amsfonts,amsmath}%,mathtext,cite,enumerate,float}

\usepackage{listings}

\usepackage{fancyhdr}
\pagestyle{fancy}

\fancyhf{}
\fancyhead[R]{\thepage}
\fancyheadoffset{0mm}
\fancyfootoffset{0mm}
\setlength{\headheight}{15mm}
%\setlength{\topmargin }{0.5mm}
\setlength{\headsep }{0 mm}
\renewcommand{\headrulewidth}{0pt}
\renewcommand{\footrulewidth}{0pt}
\fancypagestyle{plain}{ 
     \fancyhf{}
     \rhead{\thepage}
}

\sloppy{}

\makeatletter
\renewcommand*{\@biblabel}[1]{\hfill#1.}
\makeatother

\newcommand{\err}[1]{\marginpar{<=! #1}}


\usepackage{tikz}
\usetikzlibrary{positioning, arrows, shapes, snakes}

\tikzstyle{format} = [
                       thin,minimum size=1cm,
                       draw=black,
                       top color=white,
                       bottom color=white,]

\usepackage{indentfirst} % красная строка в первом абзаце
 %--------------------------------------------------------%
 \setlength{\parindent}{1.25cm}% величина абзаца

\begin{document}

\begin{titlepage}

\begin{center}
МИНОБРНАУКИ РОССИИ\\
Федеральное государственное бюджетное образовательное учреждение\\
высшего профессионального образования \\
\hspace{-10mm}"Новосибирский национальный исследовательский государственный университет"\\
(Новосибирский государственный университет)\\
\hspace{-7mm}Структурное подразделение Новосибирского государственного университета -- \\
Высший колледж информатики НГУ\\
КАФЕДРА ИНФОРМАЦИОННЫХ ТЕХНОЛОГИЙ
\end{center}


\vspace{8em}

\begin{center}
%~ \Large Пояснительная записка \\ к дипломному проекту на тему:
\Large Создания модуля для параллельного решения бигармонического уравнения методом Монте-Карло 
\end{center}

\vspace{2.5em}

\begin{center}
\textsc{Дипломный проект\\ на квалификацию техник }
\end{center}

\vspace{6em}

\begin{flushleft}
 Студент IV курса \hfill Семенов С.А. \\
%\rule{10}{1} 2013 \\
гр. 903а2 \hfill "\rule{3ex}{0,1mm}"\rule{10ex}{0,1mm}2013 \\
\vspace{1.5em}
Научный руководитель\hfill Лукинов В.Л.\\
к.ф-м.н., н.с ИВМиМГ СО РАН\hfill "\rule{3ex}{0,1mm}"\rule{10ex}{0,1mm}2013 \\
\end{flushleft}

\vspace{\fill}
\begin{center}
Новосибирск\\ 2013
\end{center}
\end{titlepage}

\setcounter{page}{2}
\tableofcontents
\newpage\likechapter{Введение}

Дипломная работа посвящена созданию эффективной библиотеки для численного решения первой краевой задачи для бигармонического уравнения методами Монте-Карло. 

Официальной датой рождения метода Монте-Карло принято считать 1949 год, когда была опубликована статья С. Улама и Н. Метрополиса \cite{int:fist} . Сам термин был предложен еще во время Второй мировой войны выдающимися учеными XX века математиком Дж. фон Нейманом и физиком Энрико Ферми в Лос-Аламосе (США) в процессе работ по ядерной тематике. Хотя методы Монте-Карло были известны и до 40-х годов, интенсивное развитие статистическое моделирование получило несколько позже в связи с появлением компьютеров, что позволило проводить вычисления больших объемов. С другой стороны, более широкое распространение получает статистическое описание тех или иных сложных физических процессов в связи с чем методы Монте-Карло все более активно используются во многих научных областях (теория переноса, теория массового обслуживания, теория надежности, статистическая физика и др.).

Основными преимуществами данных методов являются:

\begin{itemize}
	\item физическая наглядность и простота реализации,
	\item малая зависимость трудоемкости задачи от размерности,
	\item возможность решения задач со сложной геометрией,
	\item оценивание отдельных функционалов от решения без запоминания значений решения во всей области,
	\item вероятностные представления позволяют строить обобщјнные решения уравнений,
	\item одновременное оценивание вероятностной погрешности оценки искомого функционала,
	\item простое распараллеливание методов.
\end{itemize}

Бигармонические уравнения используются при решении задач теории упругости. Например, уравнение
изгиба тонких пластин имеет вид $\triangle\triangle u = g$, где $u$ --
нормальный прогиб пластины. Если пластина лежит на упругом
основании, то $u$ удовлетворяет уравнению $\triangle\triangle u
+cu=g$.

В настоящей работе использовались алгоритмы, основанные на двух принципиально разных подходах к решению краевых задач методом Монте-Карло.

Первый подход заключается в сведении исходной дифферениальной задачи к некоторому интегральному уравнению, что дает возможность использовать развитой аппарат методов Монте-Карло для решения интегральных уравнений второго рода. На этой основе строятся алгоритмы ``блуждания по сферам''.
   
Во втором подходе дифференциальная задача заменяется соответствующей разностной,
которую после приведения её к специальному виду возможно решить
методом Монте-Карло. В рамках этого подхода получаются простые и универсальные алгоритмы ``блуждания по решетке''

Несмотря на то, что рассматриваемые в дипломе алгоритмы хорошо изучены, новой и неисследованной является задача изучения данных алгоритмов при вычисления на кластерах. Основная задача дипломной работы состоит в  построении шкалируемой вычислительной библиотеки для кластерных вычислений.  

При создании требуемой библиотеки были использованы следующие программные средства и технологии.   

Message Passing Interface (MPI, интерфейс передачи сообщений) -- программный интерфейс (API)\footnote{см. сокращение} для передачи информации, который позволяет обмениваться сообщениями между процессами, выполняющими одну задачу. Разработан Уильямом Гроуппом, Эвином Ласком и другими.

MPI является наиболее распространённым стандартом интерфейса обмена данными в параллельном программировании. Существуют его реализации для большого числа компьютерных платформ. MPI используется при разработке программ для кластеров и суперкомпьютеров. Основным средством коммуникации между процессами в MPI является передача сообщений друг другу. Стандартизацией MPI занимается MPI Forum. В стандарте MPI описан интерфейс передачи сообщений, который должен поддерживаться как на платформе, так и в приложениях пользователя. В настоящее время существует большое количество бесплатных и коммерческих реализаций MPI. Существуют реализации для языков Фортран 77/90, Java, Си и Си++.

В первую очередь MPI ориентирован на системы с распределенной памятью, то есть когда затраты на передачу данных велики, в то время как OpenMP\footnote{http://openmp.org/wp/} ориентирован на системы с общей памятью (многоядерные с общим кэшем). Обе технологии могут использоваться совместно, дабы оптимально использовать в кластере многоядерные системы. Более подробно об этом \cite{mpi:offsite}.

При разработке прикладного кода использовалась распределённая система управления версиями файлов -- Git

Git — распределённая система управления версиями файлов. Проект был создан Линусом Торвальдсом для управления разработкой ядра Linux, первая версия выпущена 7 апреля 2005 года. На сегодняшний день поддерживается Джунио Хамано.

Система управления версиями (от англ. Version Control System, VCS или Revision Control System) — программное обеспечение для облегчения работы с изменяющейся информацией. Система управления версиями позволяет хранить несколько версий одного и того же документа, при необходимости возвращаться к более ранним версиям, определять, кто и когда сделал то или иное изменение, и многое другое.

Такие системы наиболее широко используются при разработке программного обеспечения для хранения исходных кодов разрабатываемой программы. Однако они могут с успехом применяться и в других областях, в которых ведётся работа с большим количеством непрерывно изменяющихся электронных документов. В частности, системы управления версиями применяются в САПР, обычно в составе систем управления данными об изделии (PDM). Управление версиями используется в инструментах конфигурационного управления (Software Configuration Management Tools).
Программа является свободной и выпущена под лицензией GNU GPL 2.

Язык для написания приложения был выбран С++ как наиболее подходящий для разработки статической библиотеки.

Статическая библиотека в программировании — сборник подпрограмм или объектов, используемых для разработки программного обеспечения (ПО) выполненых в виде исходного текста, подключаемого программистом к своей программе на этапе написания, либо в виде объектных файлов, присоединяемых (линкуемых) к исполняемой программе на этапе компиляции. В результате программа включает в себя все необходимые функции, что делает её автономной, но увеличивает размер. Без статических библиотек объектных модулей (файлов) невозможно использование большинства современных компилирующих языков и систем программирования: Fortran, Pascal, C, C++ и других.

Статическая библиотека присоединяется во время компиляции програмы в то время как присоединение динамической происходит во время выполнения.












\chapter{Постановка задачи}
\section{Конкретизации требований и задачи}
Входными условиями вычисления (пользовательскими функциями) является определение:
\begin{itemize}
	\item функции  $ \phi $;
	\item функций $ u,g $;
	\item границ области.
\end{itemize}
Функции $\phi,u,g $ соответствуют функциям в уравнении:
$ (\Delta +c)^{p+1}u=-g, (\Delta+c)^{k}u|_{\Gamma}=\phi_{k} $
Функция границ области возвращает единицу если точка с некоторой погрешностью находится на границе. 
Входными данными является:
\begin{itemize}
	\item количество путей;
	\item начальная точка.
\end{itemize}

Для задания пользовательских функций мы можем использовать программный код, прессинг функций или скрип. Первый наиболее скор в разработки, но заставляет компилировать программу каждый раз когда мы меняем вычисляемое уравнение. Для борьбы с этим недостатком сделаем вычисление в классе, который вынесем в отдельный модуль. Получаемый модуль параллельного вычисления скомпилируем как статическую библиотеку. Определение пользовательских функций проходит как задания функций обратного вызова. Так-же сделаем шаблон программы для облегчения определения пользователем своих функций. В комплект необходимо вести реализацию под конкретные условия. 

С учетом того, что конечный программный продукт будет запускается как с изменением предыдущих параметров так и для частного конкретного случая ввод данных следует сделать с помощью аргументов и(или) файлов данных.

Вывод осуществляется в tex, html файлы и на экран. 

Конкретизируем задачу: 
\begin{enumerate}
	\item Создание статической библиотеки класса с функциями обратного вызова .
	\item Создание приложение под конкретные условия.
	\item Создание файла данных под программу созданную по предыдущим условиям.
	\item Создание справки.
\end{enumerate}

Интерфейс программы смотреть приложение "Справка".

\section{Формулировка задачи}
Создание MPI приложения вычисление отклонения пластины под воздействием статичных сил.
\section{Аналоги}
%\err{qwerty uiwe rty uio wer tyu ioe rt yui o}
Главным аналогом на основе которого и разрабатывается приложение является программа Biharmon2, чей код приведен в приложении. Недостатком данной реализации алгоритмов является:
\begin{itemize}
	\item необходимость изменять алгоритм и функции основной программы(малая степень защиты от дурака);
	\item последовательность вычислений;
	\item при изменении алгоритма вычисления меняется и часть программы.
\end{itemize}
\section{блуждание по сферам}
Рассмотрим задачу Дирихле для уравнения Гельмгольца
\[ \Delta u + cu = g,	 u|_{\Gamma} = \phi \]
в области $D \subset R^{n}$ с границей $\Gamma$ , причем $c < c^{*}$, где $c^{*}$ первое собственное число оператора Лапласа для области $D, r = (x1; : : : ; xn) \in D$. Предполагаются выполненными сформулированные условия регулярности функций g, ' и границы $\Gamma$, обеспечивающие существование и единственность решения задачи , а также его вероятностное представление и интегральное представление с помощью шаровой функции Грина.

Введем следующие обозначения:\\
$\vec{D}$  - замыкание области $D$;\\
$d(P)$ - расстояние от точки $P$ до границы $\Gamma$;\\
$\epsilon > 0 $ - числовой параметр;\\
$\Gamma_{\epsilon }$ - $\epsilon$ - окрестность границы $\Gamma$, т. е. $ \Gamma_{\epsilon }=\{ P \in \vec{D}:d(P) < \epsilon \} $;\\
$S(P)$  максимальная из сфер (точнее - из гиперсфер) с центром в точке $P$, целиком лежащих в $\vec{D}, S(P) = \{Q \in \vec{D}: |Q - P| = d(P)\}$.\\
В процессе блуждания по сферам очередная точка $P_{k+1}$ выбирается равномерно по поверхности сферы $S(P_{k})$; процесс обрывается, если точка попадает в $ \Gamma_{\epsilon }$.
Дадим точное определение процесса блуждания по сферам. Зададим цепь Маркова frmgm=1;2;:::;N следующими характеристиками:
ј(r) = ±(rЎr0) - плотность начального распределения (т.е. цепь выхо-
дит из точки r0);
r; r0) = ±r(r0)  плотность перехода из r в r0
, представляющая собой
обобщенную плотность равномерного распределения вероятностей
на сфере S(r);
(r)  вероятность обрыва цепи, определяемая выражением
p0(r) =
(
0; r = 2 Γ";
1; r 2 Γ";
 номер последнего состояния.
к уже указывалось, данная цепь называется процессом блуждания
сферам. Ее можно, очевидно, записать в виде
rm = rmЎ1 + !md(rmЎ1); m = 1; 2; : : : ;
!m  последовательность независимых изотропных векторов единич-
 длины.
\section{Руководство пользователя}
\subsection{Установка}
Для сборки приложеия под Windows необходимо MPICH2, набор унтилит для компиляции: компилятор GNU GCC и GNU Make данные унтилиты представленны в пакете MinGW. Для Linux GNU GCC не обязателен, компиляция происходит силами пакета MPICH2.
\begin{enumerate}
	\item Скачайте необходимую версию библиотеки с тестовым примером.
	\item Запустите консоль в папке проэкта или перейдите в нее с помощью команды cd.
	\begin{itemize}
		\item нажмите Пуск -> Выполнить -> cmd - Это откроет консоль Windows;
		\item в консоли наберите имя диска на котором распологается проэкт с двоиточеем на конце (C:);
		\item там же напечатайте cd <путь к проэкту > (cd C:\\project).
	\end{itemize}
	\item Запустите make.
\end{enumerate}
Результатом станет скомпилированный тестовый пример и статическая библиотека находящиеся в папке с проэктом.

\subsection{Запуск приложения}
Запуск приложения описан в файле README и документации к MPICH2 

\newpage
\begin{thebibliography}{99} 
	\bibitem{int:fist}	 Ulam S. The Monte Carlo method / S. Ulam, N. Metropolis//Journal of American Statistical Association -- 1949. -- №35. -- P. 15-35.
	\bibitem{int:fty}Михайлов Г.А. Решение разностной задачи Дирихле для многомерного уравнения Гельмгольца методом Монте-Карло/ Г.А. Михайлов, А.Ф. Чешкова  // Журн. вычисл. матем. и матем. физики.  1996. -- Т. 38,  №1. -- С. 59-706.
	\bibitem{luk:dis} Лукинов В. Л. Скалярные Алгоритмы метода Монте-Карло для решения мета-гармонических уравнений: автореф. дис...канд. физ.-мат. наук / В. Л. Лукинов; ИВМиМГ СО РАН. -- Новосибирск, 2005. -- 25 с.
\end{thebibliography}

%\marginpar{просто сноска}

\appendix{}
\chapter{Диаграммы программ и исходные коды}
%\lstinputlisting[language=C++, firstline=0]{source.cpp}
\begin{figure}[hp]
\begin{small}
\begin{tikzpicture}[
                     node distance=2cm,
                     text height=1.5ex,
                     auto,
					 text width=2.7cm,
					 align=flush center
                     %text depth=.25ex
]
\node[format,rectangle,minimum size=6mm,rounded corners=3mm]  (start) {Start};
\node[format,below of=start]  (p1) {$N,U,Disp$\\$ k=0$};
\node[format,diamond,text width=1.5cm,node distance=2.5cm,aspect=2,below of=p1] (if1) {$K<N$};
\path[->] (start)  edge (p1);
\path[->] (p1)  edge (if1);


\node[format,node distance=4.5cm,text width=4cm,right of=if1]  (p2) {$Disp-=U^2$\\$ Disp=\sqrt{\frac{|Disp|}{N}}$};
\path[->] (if1)  edge node[near start] {no} (p2);
\node[format,below of=p2,text width=2cm]  (p3) {Print all};
\path[->] (p2)  edge (p3);
\node[format,rectangle,minimum size=6mm,rounded corners=3mm,below of=p3]  (stop) {End};
\path[->] (p3)  edge (stop);

\node[format,below of=if1]  (p4) {$ k++$};
\path[->] (p4)  edge (if1);

\node[format,left of=if1,node distance=5cm]  (p5) {init $x,y$\\$S,S_1=0 $};
\path[->] (if1)  edge node[near start] {yes} (p5);

\node[format,diamond,text width=2.5cm,node distance=4cm,aspect=2,below of=p5] (if2){ boundary(x,y) };
\path[->] (p5)  edge (if2);

\node[format,right of=if2,node distance=5cm,text width=5cm]  (p6) {$S += \frac{S_1}{4}\phi_1(x,y)+\phi_0(x,y)$\\$U+=\frac{S}{N}$\\$Disp+=\frac{S^2}{N}$};
\path[->] (p6)  edge (p4);
\path[->] (if2) edge node[near start] {no} (p6);

\node[format,below of=if2]  (p7) {$ d=diam(x,y)$};

\node[format,below of=p7]  (p81) {$\alpha=DRAND$\\ $\omega_1=\cos 2\pi \alpha$\\$\omega_2=\sin 2\pi \alpha$};
\node[format,below of=p81]  (p82) {$\alpha=DRAND$\\ $\omega_3=\cos 2\pi \alpha$\\$\omega_4=\sin 2\pi \alpha$};
\path[->] (if2) edge node[near start] {yes} (p7);
\path[->] (p7) edge (p81);
\path[->] (p81) edge (p82);


\node[format,right of=p82,node distance=5cm]  (p9) {$\alpha_1=DRAND$\\ $\alpha_2=DRAND$};

\node[format,diamond,text width=3cm,aspect=2,below of=p9] (if3){ $\alpha_2>(4\alpha\log\alpha_1)$ };

\path[->] (p82) edge (p9);
\path[->] (p9) edge (if3);
\path[->,draw] (if3) -- node[near start] {yes} ++(3,0) |- (p9);

\node[format,below of=if3]  (p10) {$\nu=\alpha_1d$};
\node[format,below of=p10,text width=8cm]  (p11) {$S+=S_1-d^2\frac{d^2-\nu^2-\nu^2\log{\frac{d}{\nu}}}{\log{\frac{d}{\nu}}}*\frac{g(x+\nu\omega_1,y+\nu\omega_2)}{16}$};
\node[format,below of=p11]  (p12) {$S_1-=d^2$\\$x+=\omega_1d$\\$y+=\omega_2d$};
\path[->] (if3) edge node[near start] {no} (p10);
\path[->] (p10) edge (p11);
\path[->] (p11) edge (p12);
\path[->,draw] (p12) -- ++(-8,0)  |- (if2);
\end{tikzpicture}

\end{small}
\caption{Принцип действия программы}
\end{figure}
%~ \chapter{Заголовочный файл библиотеки}
%~ \begin{spacing}{0.9}
%~ \lstinputlisting[language=C++, firstline=0,basicstyle=\small]{../gcc-con-mpi-diplom/reshotka.hpp}
%~ \end{spacing}
%\chapter{Справка}


\end{document} 
