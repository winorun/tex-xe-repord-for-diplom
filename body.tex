\chapter{Постановка задачи}
\section{Конкретизации требований и задачи}
Входными условиями вычисления (пользовательскими функциями) является определение:
\begin{itemize}
	\item функции  $ \phi $;
	\item функций $ u,g $;
	\item границ области.
\end{itemize}
Функции $\phi,u,g $ соответствуют функциям в уравнении:
$ (\Delta +c)^{p+1}u=-g, (\Delta+c)^{k}u|_{\Gamma}=\phi_{k} $
Функция границ области возвращает единицу если точка с некоторой погрешностью находится на границе. 
Входными данными является:
\begin{itemize}
	\item количество путей;
	\item начальная точка.
\end{itemize}

Для задания пользовательских функций мы можем использовать программный код, прессинг функций или скрип. Первый наиболее скор в разработки, но заставляет компилировать программу каждый раз когда мы меняем вычисляемое уравнение. Для борьбы с этим недостатком сделаем вычисление в классе, который вынесем в отдельный модуль. Получаемый модуль параллельного вычисления скомпилируем как статическую библиотеку. Определение пользовательских функций проходит как задания функций обратного вызова. Так-же сделаем шаблон программы для облегчения определения пользователем своих функций. В комплект необходимо вести реализацию под конкретные условия. 

С учетом того, что конечный программный продукт будет запускается как с изменением предыдущих параметров так и для частного конкретного случая ввод данных следует сделать с помощью аргументов и(или) файлов данных.

Вывод осуществляется в tex, html файлы и на экран. 

Конкретизируем задачу: 
\begin{enumerate}
	\item Создание статической библиотеки класса с функциями обратного вызова .
	\item Создание приложение под конкретные условия.
	\item Создание файла данных под программу созданную по предыдущим условиям.
	\item Создание справки.
\end{enumerate}

Интерфейс программы смотреть приложение "Справка".

\section{Формулировка задачи}
Создание MPI приложения вычисление отклонения пластины под воздействием статичных сил.
\section{Аналоги}
%\err{qwerty uiwe rty uio wer tyu ioe rt yui o}
Главным аналогом на основе которого и разрабатывается приложение является программа Biharmon2, чей код приведен в приложении. Недостатком данной реализации алгоритмов является:
\begin{itemize}
	\item необходимость изменять алгоритм и функции основной программы(малая степень защиты от дурака);
	\item последовательность вычислений;
	\item при изменении алгоритма вычисления меняется и часть программы.
\end{itemize}

