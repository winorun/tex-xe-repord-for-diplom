\chapter{Постановка задачи}
\section{Конкретизации требований и задачи}
Входными условиями вычисления (пользовательскими функциями) является определение:
\begin{itemize}
	\item функции  $ \phi $;
	\item функций $ u,g $;
	\item границ области.
\end{itemize}
Функции $\phi,u,g $ соответствуют функциям в уравнении:
$ (\Delta +c)^{p+1}u=-g, (\Delta+c)^{k}u|_{\Gamma}=\phi_{k} $
Функция границ области возвращает единицу если точка с некоторой погрешностью находится на границе. 
Входными данными является:
\begin{itemize}
	\item количество путей;
	\item начальная точка.
\end{itemize}

Для задания пользовательских функций мы можем использовать программный код, прессинг функций или скрип. Первый наиболее скор в разработки, но заставляет компилировать программу каждый раз когда мы меняем вычисляемое уравнение. Для борьбы с этим недостатком сделаем вычисление в классе, который вынесем в отдельный модуль. Получаемый модуль параллельного вычисления скомпилируем как статическую библиотеку. Определение пользовательских функций проходит как задания функций обратного вызова. Так-же сделаем шаблон программы для облегчения определения пользователем своих функций. В комплект необходимо вести реализацию под конкретные условия. 

С учетом того, что конечный программный продукт будет запускается как с изменением предыдущих параметров так и для частного конкретного случая ввод данных следует сделать с помощью аргументов и(или) файлов данных.

Вывод осуществляется в tex, html файлы и на экран. 

Конкретизируем задачу: 
\begin{enumerate}
	\item Создание статической библиотеки класса с функциями обратного вызова .
	\item Создание приложение под конкретные условия.
	\item Создание файла данных под программу созданную по предыдущим условиям.
	\item Создание справки.
\end{enumerate}

Интерфейс программы смотреть приложение "Справка".

\section{Формулировка задачи}
Создание MPI приложения вычисление отклонения пластины под воздействием статичных сил.
\section{Аналоги}
%\err{qwerty uiwe rty uio wer tyu ioe rt yui o}
Главным аналогом на основе которого и разрабатывается приложение является программа Biharmon2, чей код приведен в приложении. Недостатком данной реализации алгоритмов является:
\begin{itemize}
	\item необходимость изменять алгоритм и функции основной программы(малая степень защиты от дурака);
	\item последовательность вычислений;
	\item при изменении алгоритма вычисления меняется и часть программы.
\end{itemize}
\section{блуждание по сферам}
Рассмотрим задачу Дирихле для уравнения Гельмгольца
\[ \Delta u + cu = g,	 u|_{\Gamma} = \phi \]
в области $D \subset R^{n}$ с границей $\Gamma$ , причем $c < c^{*}$, где $c^{*}$ первое собственное число оператора Лапласа для области $D, r = (x1; : : : ; xn) \in D$. Предполагаются выполненными сформулированные условия регулярности функций g, ' и границы $\Gamma$, обеспечивающие существование и единственность решения задачи , а также его вероятностное представление и интегральное представление с помощью шаровой функции Грина.

Введем следующие обозначения:
\begin{itemize}
	\item $\bar{D}$  - замыкание области $D$;
	\item $d(P)$ - расстояние от точки $P$ до границы $\Gamma$;
	\item $\epsilon > 0 $ - числовой параметр;
	\item $\Gamma_{\epsilon }$ - $\epsilon$ - окрестность границы $\Gamma$, т. е. $ \Gamma_{\epsilon }=\{ P \in \bar{D}:d(P) < \epsilon \} $;
	\item $S(P)$  максимальная из сфер (точнее - из гиперсфер) с центром в точке $P$, целиком лежащих в $\bar{D}, S(P) = \{Q \in \bar{D}: |Q - P| = d(P)\}$.
\end{itemize}

В процессе блуждания по сферам очередная точка $P_{k+1}$ выбирается равномерно по поверхности сферы $S(P_{k})$; процесс обрывается, если точка попадает в $ \Gamma_{\epsilon }$. Дадим точное определение процесса блуждания по сферам. Зададим цепь Маркова $\{R_{m}\}_{m=1,2,...,N} $ следующими характеристиками:\begin{itemize}
	\item $\pi (r) = \delta(r-r_{0})$ - плотность начального распределения (т.е. цепь выходит из точки $r_0$);
	\item $p(r,r') = \delta_{r}(r') $ плотность перехода из $r$ в $r'$, представляющая собой обобщенную плотность равномерного распределения вероятностейна сфере $S(r)$;
	\item $p_{0}(r)$ вероятность обрыва цепи, определяемая выражением \[ p_{0}(r)= 
	\left\{
\begin{aligned}
& 0, r \notin \Gamma_{\epsilon}\\ & 1, r\notin \Gamma_{\epsilon}
\end{aligned}
 \right. \]
\item{}$N$ - номер последнего состояния.
\end{itemize}
Как уже указывалось, данная цепь называется процессом блуждания по сферам. Ее можно, очевидно, записать в виде $r_{m} = r_{m-1} + \omega_{m}d(r_{m-1}); m = 1; 2;....;$
$\omega_{m}$ -- последовательность независимых изотропных векторов единич-
 длины.
\section{Руководство пользователя}
\subsection{Установка}
Для сборки приложеия под Windows необходимо MPICH2, набор унтилит для компиляции: компилятор GNU GCC и GNU Make данные унтилиты представленны в пакете MinGW. Для Linux GNU GCC не обязателен, компиляция происходит силами пакета MPICH2.
\begin{enumerate}
	\item Скачайте необходимую версию библиотеки с тестовым примером.
	\item Запустите консоль в папке проэкта или перейдите в нее с помощью команды cd.
	\begin{itemize}
		\item нажмите Пуск -> Выполнить -> cmd - Это откроет консоль Windows;
		\item в консоли наберите имя диска на котором распологается проэкт с двоиточеем на конце (C:);
		\item там же напечатайте cd <путь к проэкту > (cd C:\\project).
	\end{itemize}
	\item Запустите make.
\end{enumerate}
Результатом станет скомпилированный тестовый пример и статическая библиотека находящиеся в папке с проэктом.

\subsection{Запуск приложения}
Запуск приложения описан в файле README и документации к MPICH2 
