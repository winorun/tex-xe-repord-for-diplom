\newpage\likechapter{Список условных сокращений} %обозначение и сокращения

API - Интерфейс программирования приложений (иногда интерфейс прикладного программирования) (англ. application programming interface) -- набор готовых классов, процедур, функций, структур и констант, предоставляемых приложением (библиотекой, сервисом) для использования во внешних программных продуктах. 

MPI - Message Passing Interface (интерфейс передачи сообщений) -- API для передачи информации, который позволяет обмениваться сообщениями между процессами, выполняющими одну задачу.
\newpage\likechapter{Введение}

Дипломная работа посвящена созданию эффективной библиотеки для численного решения первой краевой задачи для бигармонического уравнения методами Монте-Карло. 

Официальной датой рождения метода Монте-Карло принято считать 1949 год, когда была опубликована статья С. Улама и Н. Метрополиса \cite{int:fist} . Сам термин был предложен еще во время Второй мировой войны выдающимися учеными XX века математиком Дж. фон Нейманом и физиком Энрико Ферми в Лос-Аламосе (США) в процессе работ по ядерной тематике. Хотя методы Монте-Карло были известны и до 40-х годов, интенсивное развитие статистическое моделирование получило несколько позже в связи с появлением компьютеров, что позволило проводить вычисления больших объемов. С другой стороны, более широкое распространение получает статистическое описание тех или иных сложных физических процессов в связи с чем методы Монте-Карло все более активно используются во многих научных областях (теория переноса, теория массового обслуживания, теория надежности, статистическая физика и др.).

Основными преимуществами данных методов являются:

\begin{itemize}
	\item физическая наглядность и простота реализации,
	\item малая зависимость трудоемкости задачи от размерности,
	\item возможность решения задач со сложной геометрией,
	\item оценивание отдельных функционалов от решения без запоминания значений решения во всей области,
	\item вероятностные представления позволяют строить обобщенные решения уравнений,
	\item одновременное оценивание вероятностной погрешности оценки искомого функционала,
	\item простое распараллеливание методов.
\end{itemize}

Бигармонические уравнения используются при решении задач теории упругости. Например, уравнение
изгиба тонких пластин имеет вид $\triangle\triangle u = g$, где $u$ --
нормальный прогиб пластины. Если пластина лежит на упругом
основании, то $u$ удовлетворяет уравнению $\triangle\triangle u
+cu=g$.

В настоящей работе использовались алгоритмы, основанные на двух принципиально разных подходах к решению краевых задач методом Монте-Карло.

Первый подход заключается в сведении исходной дифференциальной задачи к некоторому интегральному уравнению, что дает возможность использовать развитой аппарат методов Монте-Карло для решения интегральных уравнений второго рода. На этой основе строятся алгоритмы "блуждания по сферам''.
   
Во втором подходе дифференциальная задача заменяется соответствующей разностной,
которую после приведения её к специальному виду возможно решить
методом Монте-Карло. В рамках этого подхода получаются простые и универсальные алгоритмы "блуждания по решетке''.

Несмотря на то, что рассматриваемые в дипломе алгоритмы хорошо изучены, новой и неисследованной является задача изучения данных алгоритмов при вычислениях на кластерах. Основная задача дипломной работы состоит в  построении библиотеки для кластерных вычислений.  












