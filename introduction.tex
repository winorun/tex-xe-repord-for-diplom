\newpage\likechapter{Введение}

Дипломная работа посвящена созданию эффективной библиотеки для численного решения первой краевой задачи для бигармонического уравнения методами Монте-Карло. 

Официальной датой рождения метода Монте-Карло принято считать 1949 год, когда была опубликована статья С. Улама и Н. Метрополиса \cite{int:fist} . Сам термин был предложен еще во время Второй мировой войны выдающимися учеными XX века математиком Дж. фон Нейманом и физиком Энрико Ферми в Лос-Аламосе (США) в процессе работ по ядерной тематике. Хотя методы Монте-Карло были известны и до 40-х годов, интенсивное развитие статистическое моделирование получило несколько позже в связи с появлением компьютеров, что позволило проводить вычисления больших объемов. С другой стороны, более широкое распространение получает статистическое описание тех или иных сложных физических процессов в связи с чем методы Монте-Карло все более активно используются во многих научных областях (теория переноса, теория массового обслуживания, теория надежности, статистическая физика и др.).

Основными преимуществами данных методов являются:

\begin{itemize}
	\item физическая наглядность и простота реализации,
	\item малая зависимость трудоемкости задачи от размерности,
	\item возможность решения задач со сложной геометрией,
	\item оценивание отдельных функционалов от решения без запоминания значений решения во всей области,
	\item вероятностные представления позволяют строить обобщјнные решения уравнений,
	\item одновременное оценивание вероятностной погрешности оценки искомого функционала,
	\item простое распараллеливание методов.
\end{itemize}

Бигармонические уравнения используются при решении задач теории упругости. Например, уравнение
изгиба тонких пластин имеет вид $\triangle\triangle u = g$, где $u$ --
нормальный прогиб пластины. Если пластина лежит на упругом
основании, то $u$ удовлетворяет уравнению $\triangle\triangle u
+cu=g$.

В настоящей работе использовались алгоритмы, основанные на двух принципиально разных подходах к решению краевых задач методом Монте-Карло.

Первый подход заключается в сведении исходной дифферениальной задачи к некоторому интегральному уравнению, что дает возможность использовать развитой аппарат методов Монте-Карло для решения интегральных уравнений второго рода. На этой основе строятся алгоритмы ``блуждания по сферам''.
   
Во втором подходе дифференциальная задача заменяется соответствующей разностной,
которую после приведения её к специальному виду возможно решить
методом Монте-Карло. В рамках этого подхода получаются простые и универсальные алгоритмы ``блуждания по решетке''

Несмотря на то, что рассматриваемые в дипломе алгоритмы хорошо изучены, новой и неисследованной является задача изучения данных алгоритмов при вычисления на кластерах. Основная задача дипломной работы состоит в  построении шкалируемой вычислительной библиотеки для кластерных вычислений.  

При создании требуемой библиотеки были использованы следующие программные средства и технологии.   

Message Passing Interface (MPI, интерфейс передачи сообщений) -- программный интерфейс (API)\footnote{см. сокращение} для передачи информации, который позволяет обмениваться сообщениями между процессами, выполняющими одну задачу. Разработан Уильямом Гроуппом, Эвином Ласком и другими.

MPI является наиболее распространённым стандартом интерфейса обмена данными в параллельном программировании. Существуют его реализации для большого числа компьютерных платформ. MPI используется при разработке программ для кластеров и суперкомпьютеров. Основным средством коммуникации между процессами в MPI является передача сообщений друг другу. Стандартизацией MPI занимается MPI Forum. В стандарте MPI описан интерфейс передачи сообщений, который должен поддерживаться как на платформе, так и в приложениях пользователя. В настоящее время существует большое количество бесплатных и коммерческих реализаций MPI. Существуют реализации для языков Фортран 77/90, Java, Си и Си++.

В первую очередь MPI ориентирован на системы с распределенной памятью, то есть когда затраты на передачу данных велики, в то время как OpenMP\footnote{http://openmp.org/wp/} ориентирован на системы с общей памятью (многоядерные с общим кэшем). Обе технологии могут использоваться совместно, дабы оптимально использовать в кластере многоядерные системы. Более подробно об этом \cite{mpi:offsite}.

При разработке прикладного кода использовалась распределённая система управления версиями файлов -- Git

Git — распределённая система управления версиями файлов. Проект был создан Линусом Торвальдсом для управления разработкой ядра Linux, первая версия выпущена 7 апреля 2005 года. На сегодняшний день поддерживается Джунио Хамано.

Система управления версиями (от англ. Version Control System, VCS или Revision Control System) — программное обеспечение для облегчения работы с изменяющейся информацией. Система управления версиями позволяет хранить несколько версий одного и того же документа, при необходимости возвращаться к более ранним версиям, определять, кто и когда сделал то или иное изменение, и многое другое.

Такие системы наиболее широко используются при разработке программного обеспечения для хранения исходных кодов разрабатываемой программы. Однако они могут с успехом применяться и в других областях, в которых ведётся работа с большим количеством непрерывно изменяющихся электронных документов. В частности, системы управления версиями применяются в САПР, обычно в составе систем управления данными об изделии (PDM). Управление версиями используется в инструментах конфигурационного управления (Software Configuration Management Tools).
Программа является свободной и выпущена под лицензией GNU GPL 2.

Язык для написания приложения был выбран С++ как наиболее подходящий для разработки статической библиотеки.

Статическая библиотека в программировании — сборник подпрограмм или объектов, используемых для разработки программного обеспечения (ПО) выполненых в виде исходного текста, подключаемого программистом к своей программе на этапе написания, либо в виде объектных файлов, присоединяемых (линкуемых) к исполняемой программе на этапе компиляции. В результате программа включает в себя все необходимые функции, что делает её автономной, но увеличивает размер. Без статических библиотек объектных модулей (файлов) невозможно использование большинства современных компилирующих языков и систем программирования: Fortran, Pascal, C, C++ и других.












